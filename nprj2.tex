\documentclass{article}
\usepackage{amsmath}
\usepackage{tikz}
\usetikzlibrary{positioning,calc,decorations.pathreplacing,shapes,fit}
\usepackage{colortbl}
\usepackage{booktabs}
\usepackage{url}
\usepackage{pifont}
\newcommand{\cmark}{\ding{51}}%
\newcommand{\xmark}{\ding{55}}%
\usepackage{array, adjustbox,url}
\usepackage{pifont,marvosym} % wasysym
\usepackage{rotating,subfig}
\usepackage{xspace}

\title{BTI 3021: Networking Project - Sprint 2}

\author{Christian Grothoff}
\date{}

\begin{document}
\maketitle

\section{Introduction}

For this sprint you will write a precursor to an IP router.  This
precursor system is to realize the ARP protocol functionality of
an IP device.

While the driver and skeleton you are given is written in C, you may
use {\em any} language of your choice for the implementation (as long
as you extend the {\tt Makefile} with adequate build rules).  However,
if you choose a different language, be prepared to write additional
boilerplate yourselves.

How the ARP protocol works is expected to be understood from the
networking class. If not, you can find plenty of documentation and
specifications on the Internet.

The basic setup is the same as in the first sprint.

\subsection{Deliverables}

There will be four main deliverables for the sprint:
\begin{description}
\item[Implementation] You must {\bf implement the ARP protocol}. Your
  implementation must answer to ARP requests, and also itself
  have the ability to issue ARP requests and to cache ARP replies.
  For this, you are to extend the {\tt arp.c} template provided
  (or write the entire logic from scratch in another language).
\item[Testing] You must implement and submit your own {\bf test cases}
  by {\em pretending} to be the network driver (see below) and sending
  ARP requests or command-line inputs to your program and verifying that it
  outputs the correct frames. Additionally, you should perform and
  {\bf document} {\em interoperability} tests against existing
  implementations (i.e. other notebooks from your team to ensure that
  your ARP protocol implementation integrates correctly with other
  implementations).
\item[Design and documentation] You must design the main
  data structure (ARP table), and create a {\bf comprehensive test plan}.
\end{description}

All deliverables must be submitted to Git (master branch)
by the submission deadline announced on Moodle.

\subsection{Functionality}

The goal is to implement a program {\tt arp} that:
\begin{enumerate}
\item Watches for ARP queries on the Ethernet link
\item Responds with ARP responses for your own IP address
\item Provides an ARP cache so that it does not have to repeatedly
  make ARP requests to the network for MAC addresses it already knows.
\item If the user just enteres ``arp'' without
  an IP address, you should output the ARP table in the format
  ``{\em IP} -$>$ {\em MAC} ({\em IFNAME})'' with one entry per line,
  i.e.
\begin{verbatim}
10.54.25.15 -> 28:c6:3f:1a:0a:bf (eth1)
\end{verbatim}
  (note the leading ``0'' digit in {\tt 0a}).
\item Reads IPv4 addresses from {\tt stdin} (in human-readable format).
  The interactive command syntax should be
  ``arp {\em IP-ADDR} {\em IFNAME}'' (i.e. each line is to be prefixed with
  the letters ``arp '', followed by the IPv4 address and the name of
  the network interface).
  \begin{enumerate}
  \item
  If the {\tt IP-ADDR} is in the ARP cache, the program must immediately
  output the associated {\em MAC}:
  \begin{verbatim}
28:c6:3f:1a:0a:bf
\end{verbatim}
  {\bf NOTE: the format requirement was clarified recently.}
  \item
  If the {\tt IP-ADDR} is {\bf not} in the ARP cache, the program should {\em only}
  issue the ARP query for those IPv4 addresses.
  \end{enumerate}
\item If an ARP response is received (at any time), the ARP cache must be
  updated accordingly (but the MAC address does not need to be output at
  that time, even if there was an explicit command-line request for this
  address before).
\end{enumerate}
Your programm should be called with the name of the interface, the IP
address\footnote{You may support multiple IPs per network interface,
  using a comma-separated list of IPs and network masks, but this is
  not required.} for that interface and the network mask.  Example:
\begin{verbatim}
$ network-driver eth0 eth1 - \
  arp eth0[IPV4:192.168.0.1/16] eth1[IPV4:10.0.0.3/24]
\end{verbatim}
This means {\tt eth0} is to be bound to 192.168.0.1 (netmask 255.255.0.0)
and {\tt eth1} uses 10.0.0.3 (netmask 255.255.255.0).

The file {\tt arp.c} provides a starting point where the parsing of
the command-line arguments and the {\tt stdin}-interaction have been
stubbed for you.


\subsection{Testing}

For your test plans, please make sure to supply the
following information for each test:

\begin{itemize}
  \item Test Case ID --- This field uniquely identifies a test case.
  \item Test case Description/Summary --- This field describes the test case objective.
  \item Pre-requisites --- This field specifies the conditions or steps that must be followed before the test steps executions.
  \item Test steps --- In this field, the exact steps are mentioned for performing the test case.
  \item Expected Results --- successful test
  \item Author --- Name of the Tester.
  \item Automation --- Whether this test case is automated or not.
  \item Status --- Pass/fail with date and code version.
\end{itemize}

Note that in addition to automated tests similar to the {\tt
  public-test-switch} you {\bf should} probably do integration tests
in a real network (evaluating manually with ping, wireshark, etc.).
This will help you find problems in your implementation and ensure
that your understanding of the network protocols is correct. Such
integration tests should be documented and will be graded as part of
the technical documentation.


\subsection{Design and Documentation}

The final documentation should include:
  \begin{itemize}
  \item Title page with group number and names/kurzel of all participants.
  \item Table of contents
  \item Introduction text (mention objectives and targeted audience)
  \item Data structure (ARP table)
  \item Test strategy
  \item Product backlog and sprint backlog (Scrum)
  \item Glossary
  \item Bibliography
\end{itemize}

Text throughout must explain methods as well as justifications for
choices made (why not other choices). The organisation should clearly
indicate sections and sub-sections.  The language should have a
consistent voice with a fairly consistent register (semi-formal).  The
goal is clarity with minimal repetition.  Concise language techniques
(relative clauses, linking words...) are expected to be employed
without forfeiting clarity.  You should also use appropriate graphical
representations and pay attention to how you visualize code.
Algorithms and data structures should be presented in diagrams or with
pseudo-code, but NOT in C syntax.

Documentation must be written in Markup and included in the Git, {\bf
  but} must also be submitted by the deadline as a PDF to the
respective Moodle assignment folder for grading by the English
professors.


\section{Grading}

{\bf All deliverables must be submitted to Git} (master branch) by the
submission deadline announced on Moodle. The PDF files generated from
the Markup documentation  must {\bf additionally be submitted via Moodle}.

You are expected to work in a team of {\bf four students}.  If needed,
Prof. Grothoff may permit the creation of a team of 5 students or
teams of 3 students.  Each team is responsible for dividing up the
work and coordinating as needed (SCRUM).

You can earn 16, 15 and 19 points in the three networking sprints, for
a total of 50 points.  You need {\bf 37 points} to pass the networking
component of BTI 3021.

If your team is {\bf for good reasons} smaller than four students, the
passing threshold will be lowered by {\bf 2 points} per ``missing''
student per sprint.  So a student going alone would still need {\bf 19
  points} to pass.

Teams may {\bf request} changes to team membership at the end of a
sprint, but must provide a {\bf justification} to the course
coordinator, who may approve or decline the request.


\subsection{ARP grading}

You get points for each of the key deliverables:
\begin{center}
\begin{tabular}{l|r}
Technical documentation (A\&D, test plan) &  2 \\ \hline
Correct implementation                    &  8 \\ \hline
Comprehensive test cases                  &  3 \\ \hline
Quality of English in documentation       &  2 \\ \hline \hline
Total                                     & 15
\end{tabular}
\end{center}

Partial points are awarded.

You can earn 16, 15 and 19 points in the three networking sprints, for
a total of 50 points.  You need {\bf 37 points} to pass the networking
component of BTI 3021.



\subsubsection{Correct implementation}
\begin{itemize}
\item 3 points for passing public test cases
\item 5 points for passing secret test cases (see Section~\ref{sec:binaries})
\end{itemize}

\subsubsection{Comprehensive test cases}
\begin{itemize}
\item 0 points if public reference implementation (see Section~\ref{sec:binaries})
      fails test cases, {\bf otherwise}
\item 1 points for failing public buggy implementation (see Section~\ref{sec:binaries}),
       plus
\item 2 points for finding bugs in secret buggy implementation(s)
\end{itemize}

If you believe you have found a bug in the provided reference
implementation, you are encouraged to discuss it with the
instructor. If you have found an actual bug in the reference
implementation (that is within the scope of the assignment), you will
be awarded a {\bf bonus point} per acknowledged bug.


\subsubsection{Technical documentation}
\begin{itemize}
\item 1 point for good description of data structure
\item 1 point for good description of test cases
\end{itemize}
Partial points may be awarded.

\subsubsection{Quality of the English in the documentation}

\begin{itemize}
\item 1 point for organisation / comprehension / consistency
\item 1 point for correct grammar / spelling / punctuation
\end{itemize}
Partial points may be awarded.



\section{Provided binaries} \label{sec:binaries}

You are provided with several binaries:
\begin{description}
\item[public-test-arp] A public test case, run using ``./public-test-arp ./arp''
  to test your switch. Returns 0 on success.
\item[public-arp] Reference implementation of the ``arp''.
\item[public-bug-arp] Buggy implementation of a ``arp``.
\end{description}


\section{Required make targets}

You may modify the build system. However, the final build system must
have the following {\tt make} targets:

\begin{description}
\item[all] build all binaries
\item[clean] remove all compiled files
\item[arp] build your ``arp`` binary from source; the binary MUST end up in the top-level directory of your build tree.
\item[test-arp] build your ``test-arp`` program from source; the program MUST end up in the top-level directory of your build tree.
\item[check-arp] Run ``test-arp`` against the ``arp'' binary.
\end{description}

For grading, we will basically run commands like:
\begin{verbatim}
GRADE=0
$ make test-arp
$ cp public-bug-arp arp
$ make check-arp || GRADE=`expr $GRADE + 1`
$ cp private-bug-arp arp
$ make check-arp || GRADE=`expr $GRADE + 2`
$ cp public-arp arp
$ make check-arp || GRADE=0
echo "Test grade: $GRADE"
\end{verbatim}
You must thus make sure the build system continues to create programs in the
right (top-level) location!


\end{document}
