\documentclass{article}
\usepackage{amsmath}
\usepackage{tikz}
\usetikzlibrary{positioning,calc,decorations.pathreplacing,shapes,fit}
\usepackage{colortbl}
\usepackage{booktabs}
\usepackage{url}
\usepackage{pifont}
\newcommand{\cmark}{\ding{51}}%
\newcommand{\xmark}{\ding{55}}%
\usepackage{array, adjustbox,url}
\usepackage{pifont,marvosym} % wasysym
\usepackage{rotating,subfig}
\usepackage{xspace}

\title{BTI 3021: Networking Project - Sprint 1}

\author{Christian Grothoff}
\date{}

\begin{document}
\maketitle

\section{Introduction}

For this sprint you will implement, document and test an Ethernet
switch in userland under GNU/Linux.

While the driver and skeleton you are given is written in C, you may
use {\em any} language of your choice for the implementation (as long
as you extend the {\tt Makefile} with adequate build rules).  However,
if you choose a different language, be prepared to write additional
boilerplate yourselves.

How an Ethernet switch works is expected to be understood from the
networking class. If not, you can find plenty of documentation and
specifications on the Internet.


\subsection{Deliverables}

There will be three main deliverables for the sprint:
\begin{description}
\item[Implementation] You must {\bf implement the switching
  algorithm}, extending the {\tt switch.c} template provided
  (or write the entire logic from scratch in another language).
\item[Testing] You must implement and submit your own {\bf test cases}
  by {\em pretending} to be the network driver (see below) and sending
  various Ethernet frames to your program and verifying that it
  outputs the correct frames. Additionally, you should perform and
  {\bf document} {\em interoperability} tests against existing
  implementations (i.e. other notebooks from your team to ensure that
  your switch integrates correctly with other implementations).
\item[Design and documentation] You must design the main
  data structure (switching table), the switching algorithm
  and create a {\bf comprehensive test plan}.
\end{description}


\subsection{Functionality}

We will specifically also look for the following properties of a switch:
\begin{itemize}
\item Support for multicast and broadcast
\item Changing external connections (re-learning when devices move around the network)
\item Managing an ``attacker'' process that sends from billions of MAC
  addresses.  Ensure your switch's learning table uses finite memory.
\end{itemize}

\subsection{Testing}

For your test plans, please make sure to supply the
following information for each test:

\begin{itemize}
  \item Test Case ID --- This field uniquely identifies a test case.
  \item Test case Description/Summary --- This field describes the test case objective.
  \item Pre-requisites --- This field specifies the conditions or steps that must be followed before the test steps executions.
  \item Test steps --- In this field, the exact steps are mentioned for performing the test case.
  \item Expected Results --- successful test
  \item Author --- Name of the Tester.
  \item Automation --- Whether this test case is automated or not.
  \item Status --- Pass/fail with date and code version.
\end{itemize}

Note that in addition to automated tests similar to the {\tt
  public-test-switch} you {\bf should} probably do integration tests
in a real network (evaluating manually with ping, wireshark, etc.).
This will help you find problems in your implementation and ensure
that your understanding of the network protocols is correct. Such
integration tests should be documented and will be graded as part of
the technical documentation.


\subsection{Design and Documentation}

The final documentation should include:
  \begin{itemize}
  \item Title page with group number and names/kurzel of all participants.
  \item Table of contents
  \item Introduction text (mention objectives and targeted audience)
  \item Data structure (switching table)
  \item Algorithm (switching algorithm) (use UML state diagram)
  \item Test strategy
  \item Product backlog and sprint backlog (Scrum)
  \item Glossary
  \item Bibliography
\end{itemize}

Text throughout must explain methods as well as justifications for
choices made (why not other choices). The organisation should clearly
indicate sections and sub-sections.  The language should have a
consistent voice with a fairly consistent register (semi-formal).  The
goal is clarity with minimal repetition.  Concise language techniques
(relative clauses, linking words...) are expected to be employed
without forfeiting clarity.  You should also use appropriate graphical
representations and pay attention to how you visualize code.
Algorithms and data structures should be presented in diagrams or with
pseudo-code, but NOT in C syntax.

Documentation must be written in Markup and included in the Git, {\bf
  but} must also be submitted by the deadline as a PDF to the
respective Moodle assignment folder for grading by the English
professors.


\section{Grading}

{\bf All deliverables must be submitted to Git} (master branch) by the
submission deadline announced on Moodle. The PDF files generated from
the Markup documentation must {\bf additionally be submitted via Moodle}.

You are expected to work in a team of {\bf four students}.  If needed,
Prof. Grothoff may permit the creation of a team of 5 students or
teams of 3 students.  Each team is responsible for dividing up the
work and coordinating as needed (SCRUM).

You can earn 16, 15 and 19 points in the three networking sprints, for
a total of 50 points.  You need {\bf 37 points} to pass the networking
component of BTI 3021.

If your team is {\bf for good reasons} smaller than four students, the
passing threshold will be lowered by {\bf 2 points} per ``missing''
student per sprint.  So a student going alone would still need {\bf 19
  points} to pass.

Teams may {\bf request} changes to team membership at the end of a
sprint, but must provide a {\bf justification} to the course
coordinator, who may approve or decline the request.



\subsection{Switch grading}

For the {\bf switch} sprint, you get points for each of the key deliverables:
\begin{center}
\begin{tabular}{l|r}
Correct implementation                    &  7 \\ \hline
Comprehensive test cases                  &  4 \\ \hline
Technical documentation (A\&D, test plan) &  3 \\ \hline
Quality of English in documentation       &  2 \\ \hline \hline
Total                                     & 16
\end{tabular}
\end{center}

\subsubsection{Correct implementation}
\begin{itemize}
\item 3 points for passing public test cases
\item 4 points for passing secret test cases (see Section~\ref{sec:binaries})
\end{itemize}

\subsubsection{Comprehensive test cases}
\begin{itemize}
\item 0 points if public reference implementation (see Section~\ref{sec:binaries})
      fails test cases, {\bf otherwise}
\item 2 points for failing public buggy implementation (see Section~\ref{sec:binaries}),
       plus
\item 2 points for finding bugs in secret buggy implementation(s)
\end{itemize}

If you believe you have found a bug in the provided reference
implementation, you are encouraged to discuss it with the
instructor. If you have found an actual bug in the reference
implementation (that is within the scope of the assignment), you will
be awarded a {\bf bonus point} per acknowledged bug.


\subsubsection{Technical documentation}
\begin{itemize}
\item 1 point for good description of data structure
\item 1 point for good description of switching algorithm
\item 1 point for good description of test cases
\end{itemize}
Partial points may be awarded.

\subsubsection{Quality of the English in the documentation}

\begin{itemize}
\item 1 point for organisation / comprehension / consistency
\item 1 point for correct grammar / spelling / punctuation
\end{itemize}
Partial points may be awarded.



\section{Hardware}

You will be given a 4-port Ethernet USB adapter that you can use to
add four physical ports to any PC.  If you use the laboratory PCs, be
aware that some of the USB ports provide insufficient power. Which
ones work is inconsistent even across identical PCs and often even the
adjacent USB port works even though it looks identical!

You are not expected to write a driver to interact directly with the
Ethernet USB adapter.  Instead, you will use the provided
{\tt network-driver} which can already provide you with raw access
to any Ethernet interface (incl. WLAN).

\subsection{Alternative setup with virtual machines}


Clone the Git repository at
\url{https://gitlab.ti.bfh.ch/demos/vlab} and follow the provided
instructions.

\section{The {\tt network-driver}}

To access the hardware, your final program should be {\em executed}
by the {\tt network-driver}.  For this, you call
\begin{verbatim}
$ network-driver IFC1 ... IFCn - ./switch ARGS
\end{verbatim}
where ``IFC1 ... IFCn'' is the list of interface names that you want
{\tt network-driver} to support (i.e. ``lan0'', ``lan1'') and ``PROG''
is the name of your binary and ``ARGS'' are the command-line arguments
to ``switch''.  Note the ``-'' (single minus) between the last
interface name and ``switch''.  Also, ``./switch'' must be given with
its path (i.e. ``./switch'' for the current working directory) or be
located in a directory that is given in the ``PATH'' environment
variable.

Once you start {\tt switch} like this, you can read Ethernet frames
and end-user commands from ``stdin'' and write Ethernet
frames (and end-user output) to ``stdout''.

Note that you must follow the {\tt network-driver}'s particular
format for inter-process communication when reading and writing.
You will {\bf not} be communicating directly with the console!


\subsection{Build the driver}

To compile the code, run:
\begin{verbatim}
# This requires gcc
$ make
# Creating network interfaces requires 'root' rights
$ sudo chmod +s network-driver
# Try it out:
$ ./network-driver eth0 - ./parser
\end{verbatim}
Press CTRL-C to stop the {\tt network-driver} and {\tt parser}.


\subsection{Understanding the driver}

The output of the driver is always in binary and generally in network
byte order.  You can use a tool like {\tt hexer} to make the output
slightly more readable.

The driver will always output a series of messages starting with
a {\tt struct GLAB\_MessageHeader} that includes a type and a size.

When the driver starts, it first writes a control message (of type 0)
with payload that includes 6 bytes for each of the local interface's
MAC addresses to your {\tt stdin}.  Henceforce, messages received
of type 0 will be single lines of command-line input (including the
'\\n'-terminator, but excluding the 0-terminator of C) as typed in
by the user.

Furthermore, the driver will output a {\tt struct GLAB\_MessageHeader}
for each frame received.  The {\tt struct GLAB\_MessageHeader} will be
followed by the actual network frame, starting with the Ethernet frame
excluding preamble, delimiter and FCS.  The {\tt struct
  GLAB\_MessageHeader} includes the total length of the subsequent
frame (encoded in network byte order, the size includes the {\tt
  struct GLAB\_MessageHeader}).  The fixed message type identifies the
number of the network interface, counting from one (also in network
byte order).

In addition to writing received frames to your {\tt stdin}, the driver
also tries to read from your {\tt stdout}.  Applications must send the
same message format to {\tt stdout} that the driver sends them on {\tt
  stdin}.  The driver does {\bf not} check that the source MAC is set
correctly!

To write to the console's {\tt stdout}, use a message type of 0.
You may directly write to {\tt stderr} for error messages.


\section{Provided code}

You are given a few C snippets as starting points. However, these
mostly serve to {\em illustrate} how to process the output from the
driver. You are completely free to implement your application in {\em
  any} programming language.  Note that each file includes about 20
LOC of a licensing statement, so the functions provided should not
provide a significant advantage for implementations in C.

\begin{description}
\item[sample-parser.c]{This file includes a simple starting point for
  a wireshark-like frame inspection code.
  It mostly shows how the frames are received and
  a bit how to use the other C files. (82 LOC)}
\item[glab.h]{A struct defining a MAC Address and a few common C includes. (90 LOC)}
\item[print.c]{This file shows how to wrap messages to print them
  via the driver. (112 LOC)}
\item[loop.c]{This could be the main loop of your application. The code includes
  some basic logic that looks at each frame, decides whether it is the MACs,
  control or an Ethernet frame and then calls the respective function. (93 LOC)}
\item[crc.c]{An implementation of checksum algorithms. (194 LOC)}
\end{description}

If you are using another programming language, you are free to re-use
an existing CRC implementation in that language.


The main file for the exercise is {\tt switch.c}. In this file, you should
implement a program {\tt switch} which forwards frames received on any
interface to any other interface, but passively learns MAC addresses
and optimizes subsequent traffic.


\section{Provided binaries} \label{sec:binaries}

You are provided with several binaries:
\begin{description}
\item[public-test-switch] A public test case, run using ``./public-test-switch ./switch''
  to test your switch. Returns 0 on success.
\item[public-switch] Reference implementation of the ``switch''.
\item[public-bug-switch] Buggy implementation of a ``switch``.
\end{description}


\section{Required make targets}

You may modify the build system. However, the final build system must
have the following {\tt make} targets:

\begin{description}
\item[all] build all binaries
\item[clean] remove all compiled files
\item[switch] build your ``switch`` binary from source; the binary MUST end up in the top-level directory of your build tree.
\item[test-switch] build your ``test-switch`` program from source; the program MUST end up in the top-level directory of your build tree.
\item[check-switch] Run ``test-switch`` against the ``switch'' binary.
\end{description}

For grading, we will basically run commands like:
\begin{verbatim}
GRADE=0
$ make test-switch
$ cp public-bug-switch switch
$ make check-switch || GRADE=`expr $GRADE + 2`
$ cp private-bug-switch switch
$ make check-switch || GRADE=`expr $GRADE + 2`
$ cp public-switch switch
$ make check-switch || GRADE=0
echo "Test grade: $GRADE"
\end{verbatim}
You must thus make sure the build system continues to create programs in the
right (top-level) location!


\section{Warm-up exercises}

These exercises may help you get familiar with the provided environment.
They are {\bf not graded}.

\subsection{Frame parsing}

Extend the simple frame {\tt parser} to:
\begin{enumerate}
\item Output your system's MAC address(es) in the canonical human-readable format.
\item Output the source MAC, destination MAC, payload type and payload length
  of each frame received. Confirm your results with {\tt wireshark}.
\end{enumerate}

\subsection{Implement a Hub}

Implement {\tt hub} which forwards frames received on any interface to
all other interfaces ({\tt eth0} through {\tt eth3}), without changing
them at all.  The {\tt hub} binary should take the list of interfaces
to use on the command line.


\section{Bonus exercise}

This exercise may help you better understand the material from the
networking course. It is not expected to be achievable within the
period of the sprint and best done as a follow-up. It is {\bf not
  graded}.

Implement {\em vswitch} which forwards frames received on any
interface to any other interface, passively learns MAC addresses,
and respects VLAN tags. As before, the command-line specifies the
list of network interfaces you should switch on, but with
additional options to specify the VLANS.  Example:
\begin{verbatim}
$ network-driver eth0 eth1 eth2 eth3 - \
  vswitch eth0[T:1,2] eth1[U:1] eth2[U:2] eth3[U:2]
\end{verbatim}
This is supposed to run VLANs 1 and 2 tagged on {\tt eth0},
and VLANs 1, 2 or 2 untagged on {\tt eth1}, {\tt eth2},
or {\tt eth3} respectively.  Network interfaces specified
without ``[]'' should operate untagged on VLAN 0.  It is
not allowed to have interfaces accept both tagged and
untagged frames.

Test your implementation against the Netgear switch of the lab.
Bridge a tagged VLAN ({\tt VID}$=3$) from the Netgear switch ({\tt
  eth1}) with two untagged notebooks ({\tt eth2}, {\tt eth3}) using
the BananaPi.

Properties to consider for a virtual switch:
\begin{itemize}
\item Correct forwarding? Do frames flow bidirectionally between eth2 and eth3?
\item Correct switching? Is learning correctly implemented?
\item Are VLAN tags added/stripped?
\item Are correct limitations imposed on VLANs?
  For example using the configuration above, you might check that
  frames with $VID=2$ are not forwarded to eth1.
\end{itemize}


\end{document}


\subsection{Alternative setup with qemu}



Download a basic qemu VM, for example from
\url{https://people.debian.org/~aurel32/qemu/amd64/}.

Then launch your router with four network interfaces and
the respective up scripts and disjoint MACs:
\begin{verbatim}
# qemu-system-x86_64 -hda wheezy.qcov2 \
   -device e1000,netdev=net0,mac=10:10:A0:D0:C0:B1 \
   -netdev socket,id=net0,listen=:10001 \
   -device e1000,netdev=net1,mac=10:10:A0:D0:C0:B2 \
   -netdev socket,id=net0,listen=:10002 \
   -device e1000,netdev=net2,mac=10:10:A0:D0:C0:B3 \
   -netdev socket,id=net0,listen=:10003 \
   -device e1000,netdev=net3,mac=10:10:A0:D0:C0:B4 \
   -netdev socket,id=net0,listen=:10004
\end{verbatim}
Then, launch additional VMs to connect ``other'' systems
to your router:
\begin{verbatim}
# cp wheezy.qcov2 wheezy0.qcov
# qemu-system-x86_64 -hda wheezy0.qcov2 \
   -device e1000,netdev=net0,mac=10:20:A0:D0:C0:B1 \
   -netdev socket,id=net0,connect:10001
# cp wheezy.qcov2 wheezy1.qcov
# qemu-system-x86_64 -hda wheezy1.qcov2 \
   -device e1000,netdev=net0,mac=10:20:A0:D0:C0:B2 \
   -netdev socket,id=net0,connect:10002
# cp wheezy.qcov2 wheezy2.qcov
# qemu-system-x86_64 -hda wheezy2.qcov2 \
   -device e1000,netdev=net0,mac=10:20:A0:D0:C0:B3 \
   -netdev socket,id=net0,connect:10003
# cp wheezy.qcov2 wheezy3.qcov
# qemu-system-x86_64 -hda wheezy3.qcov2 \
   -device e1000,netdev=net0,mac=10:20:A0:D0:C0:B4 \
   -netdev socket,id=net0,connect:10004
\end{verbatim}
You may want to link up your VMs to your host to copy files:
\begin{verbatim}
# qemu-system-x86_64 ...
  -net nic -net user,hostfwd=tcp::5555-:22
\end{verbatim}
Now your localhost port 5555 is forwarded to port 22 of the
virtual machine.
