\documentclass{article}
\usepackage{amsmath}
\usepackage{tikz}
\usetikzlibrary{positioning,calc,decorations.pathreplacing,shapes,fit}
\usepackage{colortbl}
\usepackage{booktabs}
\usepackage{url}
\usepackage{pifont}
\newcommand{\cmark}{\ding{51}}%
\newcommand{\xmark}{\ding{55}}%
\usepackage{array, adjustbox,url}
\usepackage{pifont,marvosym} % wasysym
\usepackage{rotating,subfig}
\usepackage{xspace}

\title{BTI 3021: Networking Project - Sprint 3}

\author{Christian Grothoff}
\date{}

\begin{document}
\maketitle

\section{Introduction}

For this sprint you will write an IP router, building upon the
results from your sprint.

While the driver and skeleton you are given is written in C, you may
again use {\em any} language of your choice for the implementation (as
long as you extend the {\tt Makefile} with adequate build rules).
However, if you choose a different language, be prepared to write
additional boilerplate yourselves.

How an IP router works is expected to be understood from the
networking class. If not, you can find plenty of documentation and
specifications on the Internet.

The basic setup is the same as in the first two sprints.

\subsection{Deliverables}

There will be three main deliverables for the sprint:
\begin{description}
\item[Implementation] You must implement an IPv4 router. Your
  implementation must answer to IP packets, and route them.
  For this, you are to extend the {\tt router.c} template provided
  (or write the entire logic from scratch in another language).
\item[Testing] You must implement and submit your own {\bf test cases}
  by {\em pretending} to be the network driver (see below) and sending
  IP packets or command-line inputs to your program and verifying that it
  outputs the correct frames. Additionally, you should perform and
  {\bf document} {\em interoperability} tests against existing
  implementations (i.e. other notebooks from your team to ensure that
  your IP router implementation integrates correctly with other
  implementations).
\item[Design and documentation] You must design the main
  data structure (routing table), the routing and fragmentation algorithms
  and create a {\bf comprehensive test plan}.
\end{description}


\subsection{Functionality}

Implement {\tt router} which routes IPv4 packets:
\begin{enumerate}
\item Populate your routing table from the network interface configuration
  given on the command-line using the same syntax as with the {\tt arp}
  program.
\item Use the ARP logic to resolve the target MAC address.   Simply drop the IP
  packets for destinations where the next hop's MAC address has not yet been
  learned, and issue the ARP request to obtain the destination’s MAC instead
  (once per dropped IP packet).
\item Make sure to decrement the TTL field and recompute the CRC.
  % add link to logic implementing CRC?
\item Generate ICMP messages for ``no route to network'' (ICMP
    Type 3, Code 0) and ``TTL exceeded'' (ICMP Type 11, Code 0),
\item Support the syntax {\tt IFC[RO]=MTU} where {\tt MTU} is the
  MTU for IFC.  Example: {\tt eth0=1500}.  Implement and test IPv4 fragmentation
  (including {\em do not fragment}-flag support), including sending
  ICMP  (ICMP Type 3, Code 4).
\item Support dynamic updates to the routing table via {\tt stdin}.
  Base your commands on the {\tt ip route} tool.  For example,
  ``route list'' should output the routing table, and
  ``route add 1.2.0.0/16 via 192.168.0.1 dev eth0'' should add
  a route to {\tt 1.2.0.0/16} via the next hop {\tt 192.168.0.1}
  which should be reachable via {\tt eth0}.  Implement at least
  the {\tt route list}, {\tt route add} and {\tt route del} commands.
  The interface-specific (connected local network) routes that
  are added upon startup from the command-line must not need to be
  {\tt del}etable.
\end{enumerate}

The output of your routing table should have the following format:
\begin{verbatim}
192.168.0.0/255.255.0.0 -> 1.2.3.4 (eth0)
\end{verbatim}
Use 0.0.0.0 if there is no next hop (the target host is in the connected
LAN on the specified interface).  You may print the routing table
in any order. Do include locally connected networks.

Routing table entries for locally connected networks MUST NOT be
configured explictly (via ``route add``) but must be automatically
created when your router starts (from the command-line arguments). You
do not have to support removal of those routing table entries.


Note that your implementation must realize following functions of a
router:

\begin{itemize}
\item Basic IP handling (TTL, ICMP, Checksum) % IP TTL decremented, ICMP? Checksum?
\item Forwarding and routing % IP packets flow? Mac updated?
\item Address resultion and caching % ARP cache?
\item IP fragmentation % Use eth3 for testing
\end{itemize}


\subsection{Testing}

For your test plans, please make sure to supply the
following information for each test:

\begin{itemize}
  \item Test Case ID --- This field uniquely identifies a test case.
  \item Test case Description/Summary --- This field describes the test case objective.
  \item Pre-requisites --- This field specifies the conditions or steps that must be followed before the test steps executions.
  \item Test steps --- In this field, the exact steps are mentioned for performing the test case.
  \item Expected Results --- successful test
  \item Author --- Name of the Tester.
  \item Automation --- Whether this test case is automated or not.
  \item Status --- Pass/fail with date and code version.
\end{itemize}

Note that in addition to automated tests similar to the {\tt
  public-test-switch} you {\bf should} probably do integration tests
in a real network (evaluating manually with ping, wireshark, etc.).
This will help you find problems in your implementation and ensure
that your understanding of the network protocols is correct. Such
integration tests should be documented and will be graded as part of
the technical documentation.


\subsection{Design and Documentation}

The final documentation should include:
  \begin{itemize}
  \item Title page with group number and names/kurzel of all participants.
  \item Table of contents
  \item Introduction text (mention objectives and targeted audience)
  \item Data structure (routing table)
  \item Routing algorithm (use UML state diagram)
  \item Fragmentation algorithm (use UML state diagram)
  \item Test strategy
  \item Product backlog and sprint backlog (Scrum)
  \item Glossary
  \item Bibliography
\end{itemize}

Text throughout must explain methods as well as justifications for
choices made (why not other choices). The organisation should clearly
indicate sections and sub-sections.  The language should have a
consistent voice with a fairly consistent register (semi-formal).  The
goal is clarity with minimal repetition.  Concise language techniques
(relative clauses, linking words...) are expected to be employed
without forfeiting clarity.  You should also use appropriate graphical
representations and pay attention to how you visualize code.
Algorithms and data structures should be presented in diagrams or with
pseudo-code, but NOT in C syntax.

Documentation must be written in Markup and included in the Git, {\bf
  but} must also be submitted by the deadline as a PDF to the
respective Moodle assignment folder for grading by the English
professors.



\section{Grading}

{\bf All deliverables must be submitted to Git} (master branch) by the
submission deadline announced on Moodle. The PDF files generated from
the Markup documentation must {\bf additionally be submitted via Moodle}.

You are expected to work in a team of {\bf four students}.  If needed,
Prof. Grothoff may permit the creation of a team of 5 students or
teams of 3 students.  Each team is responsible for dividing up the
work and coordinating as needed (SCRUM).

You can earn 16, 15 and 19 points in the three networking sprints, for
a total of 50 points.  You need {\bf 37 points} to pass the networking
component of BTI 3021.

If your team is {\bf for good reasons} smaller than four students, the
passing threshold will be lowered by {\bf 2 points} per ``missing''
student per sprint.  So a student going alone would still need {\bf 19
  points} to pass.

Teams may {\bf request} changes to team membership at the end of a
sprint, but must provide a {\bf justification} to the course
coordinator, who may approve or decline the request.

\subsection{Router grading}

You get points for each of the key deliverables:
\begin{center}
\begin{tabular}{l|r}
Correct implementation                    & 10 \\ \hline
Comprehensive test cases                  &  4 \\ \hline
Technical documentation (A\&D, test plan) &  3 \\ \hline
Quality of English in documentation       &  2 \\ \hline \hline
Total                                     & 19
\end{tabular}
\end{center}


\subsubsection{Correct implementation}
\begin{itemize}
\item 3 points for passing public test cases
\item 7 points for passing secret test cases (see Section~\ref{sec:binaries})
\end{itemize}

\subsubsection{Comprehensive test cases}
\begin{itemize}
\item 0 points if public reference implementation (see Section~\ref{sec:binaries})
      fails test cases, {\bf otherwise}
\item 2 points for failing public buggy implementation (see Section~\ref{sec:binaries}),
       plus
\item 2 points for finding bugs in secret buggy implementation(s)
\end{itemize}

If you believe you have found a bug in the provided reference
implementation, you are encouraged to discuss it with the
instructor. If you have found an actual bug in the reference
implementation (that is within the scope of the assignment), you will
be awarded a {\bf bonus point} per acknowledged bug.


\subsubsection{Technical documentation}
\begin{itemize}
\item 1 point for good description of data structure
\item 1 point for good description of switching algorithm
\item 1 point for good description of test cases
\end{itemize}
Partial points may be awarded.

\subsubsection{Quality of the English in the documentation}

\begin{itemize}
\item 1 points for organisation / comprehension / consistency
\item 1 points for correct grammar / spelling / punctuation
\end{itemize}
Partial points may be awarded.



\section{Provided binaries} \label{sec:binaries}

You are provided with several binaries:
\begin{description}
\item[public-test-router] A public test case, run using ``./public-test-router ./router''
  to test your router. Returns 0 on success.
\item[public-router] Reference implementation of the ``router''.
\item[public-bug-router] Buggy implementation of a ``router``.
\end{description}


\section{Required make targets}

You may modify the build system. However, the final build system must
have the following {\tt make} targets:

\begin{description}
\item[all] build all binaries
\item[clean] remove all compiled files
\item[router] build your ``router`` binary from source; the binary MUST end up in the top-level directory of your build tree.
\item[test-router] build your ``test-router`` program from source; the program MUST end up in the top-level directory of your build tree.
\item[check-router] Run ``test-router`` against the ``router'' binary.
\end{description}

For grading, we will basically run commands like:
\begin{verbatim}
GRADE=0
$ make test-router
$ cp public-bug-router router
$ make check-router || GRADE=`expr $GRADE + 2`
$ cp private-bug-router router
$ make check-router || GRADE=`expr $GRADE + 2`
$ cp public-router router
$ make check-router || GRADE=0
echo "Test grade: $GRADE"
\end{verbatim}
You must thus make sure the build system continues to create programs in the
right (top-level) location!



% You do not need to support VLANs, IP multicast or IP broadcast.


%Configure your router with {\tt eth1} using 192.168.0.1/16.  Configure
%{\tt eth2} using 10.0.0.1/8 and {\tt eth3} using 172.16.0.1/12.
%Connect your notebook as 10.0.0.2 using 10.0.0.1 as the default
%route. Set an MTU of 500 on {\tt eth3}.  Set a default route of
%192.168.0.2 on the router.


\end{document}
