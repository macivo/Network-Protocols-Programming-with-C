\documentclass{article}
\usepackage{amsmath}
\usepackage{tikz}
\usetikzlibrary{positioning,calc,decorations.pathreplacing,shapes,fit}
\usepackage{colortbl}
\usepackage{booktabs}
\usepackage{url}
\usepackage{hyperref}
\usepackage{pifont}
\newcommand{\cmark}{\ding{51}}%
\newcommand{\xmark}{\ding{55}}%
\usepackage{array, adjustbox,url}
\usepackage{pifont,marvosym} % wasysym
\usepackage{rotating,subfig}
\usepackage{xspace}
\usepackage{tikz}
\usetikzlibrary{positioning,calc,decorations.pathreplacing,shapes}

\title{GLab: Frequenty Asked Questions}

\author{Christian Grothoff, Hansjürg Wenger}
\date{\today}

\begin{document}
\maketitle


\section{C Programming}

\subsection{How do I print a MAC address?}

Use something like this:
\begin{verbatim}
static void
print_mac (const struct MacAddress *mac)
{
  print ("%02X:%02X:%02X:%02X:%02X:%02X\n
         mac->mac[0], mac->mac[1],
         mac->mac[2], mac->mac[3],
         mac->mac[4], mac->mac[5]);
}
\end{verbatim}

\subsection{How do I compare two MAC addresses?}

Use something like this:
\begin{verbatim}
static int
maccmp (const struct MacAddress *mac1,
        const struct MacAddress *mac2)
{
  return memcmp (mac1, mac2, sizeof (struct MacAddress));
}
\end{verbatim}

\subsection{How do I print an IPv4 address?}

Use something like this:
\begin{verbatim}
#include <arpa/inet.h>

static void
print_ip (const struct in_addr *ip)
{
  char buf[INET_ADDRSTRLEN];
  print ("%s\n
         inet_ntop (AF_INET,
                    ip,
                    buf,
                    sizeof (buf)));
}
\end{verbatim}


\subsection{How do pointers work again?}

Consider this code:
\begin{verbatim}
int i = 5;
int *j = &i;
int k = *j;
\end{verbatim}
$i$ is an integer. When you write just $i$, you will get the value
$5$. $i$ is stored in computer memory. When you write $\&i$, you get
the address where $i$ is stored in memory.  $j$ is a pointer to
an integer. So the value of $j$ is an address in memory where we
expect to find an integer.  If we want to read the integer at that
address, we write $*j$ to dereference the address $j$.


\subsection{How do arrays work again?}

Arrays in C are represented as pointers to the first element of the
array.

Consider this code:
\begin{verbatim}
int i[] = { 5 , 6 };
int *j = i;
i[0] + j[1]; // 11
\end{verbatim}
Here $i$ is declared as an array with two values, 5 and 6.  $j$ is,
like $i$, a pointer.  You can treat pointers like arrays and vice-versa.
The only difference is that {\tt sizeof(i)} will give you the size of
the array as the C compiler knows how large the array is in memory, while
{\tt sizeof(j)} will give you the size of (any) pointer.

The following code generates exactly the same situation as the fragment
above:
\begin{verbatim}
int i[2];
int *j = i;
*i = 5;
*(j+1) = 6;
\end{verbatim}


\subsection{How do I get a pointer to the IPv4 header?}


A slightly unclean\footnote{due to unaligned pointer access}
minimalistic solution that works fine on your CPUs would look like
this:
\begin{verbatim}
static void
handle_frame (uint16_t interface,
	      const void *frame,
	      size_t frame_size)
{
  struct EthernetHeader *eh = frame;
  struct IPv4Header *ip = (struct IPv4Header *) &frame[1];

  if (frame_size < sizeof (struct EthernetHeader) +
                   sizeof (struct IPv4Header))
    fail ();
  // use(ip); here!
}
\end{verbatim}



\subsection{Why does my {\tt printf} not work?}

{\tt printf} prints to {\tt stdout}, which in your setup is the
network-driver which expects to receive frames to be transmitted.
To output to the console, you need to give a special command to
the network-driver.  This is provided by the {\tt print()} function,
which thus replaces {\tt printf} for your project.  You can
also directly use {\tt fprintf (stderr, ...)} to write to {\tt stderr}
for logging. Note that required output to the user (for grading) must
be generated using {\tt print}.

\subsection{I am using {\tt print}, and it still does not work!}

You might be having a problem with terminal discipline. {\tt print}ing
to {\tt stdout} is by default buffered until a newline is encountered.
Make sure to terminate your output with ``\\n''.


\newpage
\section{Setup}

\subsection{Why do I get ``permission denied'' when running the network-driver, even as root?}

The latest version of the Ubuntu operating system refuses
to grant RAW socket access to binaries that are in the {\tt /home}
directory, even if run by {\tt root}.  Copying the binary to
{\tt /root} or {/tt /usr/sbin} makes it work:

\begin{verbatim}
# cp network-driver /usr/sbin/network-driver
# /usr/sbin/network-driver eth0 - ./parser
\end{verbatim}

\subsection{What should my network topology look like?}

Figure~\ref{fig:setup} shows the suggested setup.  If you are a team
of three students, you may want to connect a third notebook. If you
are using the Banana PI, you must use Ethernet USB adapters instead of
the Ethernet ports of the PI for the connections to the notebooks for
the {\tt hub}.  {\tt switch} and {\tt vswitch} programs.

\begin{figure}[h!]
  \centering

\colorlet{FGcolor}{green}

\colorlet{FG100P}{FGcolor!100}
\colorlet{FG90P}{FGcolor!90}
\colorlet{FG80P}{FGcolor!80}
\colorlet{FG70P}{FGcolor!70}
\colorlet{FG60P}{FGcolor!60}
\colorlet{FG50P}{FGcolor!50}
\colorlet{FG40P}{FGcolor!40}
\colorlet{FG30P}{FGcolor!30}
\colorlet{FG20P}{FGcolor!20}
\colorlet{FG10P}{FGcolor!10}
\colorlet{FG00P}{FGcolor!00}

% Use RGB: TUMBlue, TUMDarkBlue, TUMDarkerBlue
\definecolor{color_bg}{rgb}{0,0,1}

\tikzstyle{bg_fill_10} = [fill=FG10P]
\tikzstyle{bg_fill_20} = [fill=FG20P]
\tikzstyle{bg_fill_30} = [fill=FG30P]
\tikzstyle{bg_fill_40} = [fill=FG40P]
\tikzstyle{bg_fill_50} = [fill=FG50P]
\tikzstyle{bg_fill_60} = [fill=FG60P]
\tikzstyle{bg_fill_70} = [fill=FG70P]
\tikzstyle{bg_fill_80} = [fill=FG80P]
\tikzstyle{bg_fill_90} = [fill=FG90P]
\tikzstyle{bg_fill_100} = [fill=FG100P]

% Define Sizes
\newcommand\defaultpicheight{2em}
\newcommand\defaultpicheightsmall{1em}
\newcommand\defaultblockheight{1.5em}
\newcommand\defaultblockwidth{5cm}
\newcommand\defaultblockdistance{0.25em}
\newcommand\defaultinterblockdistance{9em}
\newcommand\dibd{\defaultinterblockdistance}
\newcommand\dbh{\defaultblockheight}
\newcommand\blocksizedecrease{0.95}
\newcommand\blocksizeincrease{1.05}

\newcommand*\circled[1]{\tikz[baseline=(char.base)]{
            \node[shape=circle,minimum width = 2em, draw,inner sep=2pt] (char) {#1};}}

% Define Blocks
\tikzstyle{my_block_width} = [minimum width=\defaultblockwidth]
\tikzstyle{my_block_height} = [minimum height=\defaultblockheight]

% Cliparts
%\pgfdeclareimage[height=\defaultpicheight]{webserver}{cliparts/webserver}
%\pgfdeclareimage[height=\defaultpicheight]{house}{cliparts/house}
%\pgfdeclareimage[height=\defaultpicheight]{router}{cliparts/router}
%\pgfdeclareimage[height=\defaultpicheight]{femaleuser}{cliparts/femaleuser}
%\pgfdeclareimage[height=\defaultpicheight]{maleuser}{cliparts/maleuser}
%\pgfdeclareimage[height=\defaultpicheight]{globe}{cliparts/globe}
%\pgfdeclareimage[height=\defaultpicheight]{document}{cliparts/document}
%\pgfdeclareimage[height=\defaultpicheightsmall]{documentsmall}{cliparts/document}
%\pgfdeclareimage[height=\defaultpicheight]{certificate}{cliparts/certificate}

% Block style
\tikzstyle{my_block} = [draw, rectangle, align=center, semithick, rounded corners,
  my_block_height, my_block_width]

\tikzstyle{my_dashed_block} = [draw, rectangle, align=center, semithick, dashed, rounded corners,
  my_block_height, my_block_width]

\tikzstyle{my_inv_block} = [draw, rectangle, align=center, draw=none,
  my_block_height, my_block_width]

\tikzstyle{my_text_block} = [draw, rectangle, align=center, draw=none]

\begin{tikzpicture}[auto, node distance=5cm,>=latex]
	\node [my_block, bg_fill_20,
				minimum width =  \defaultblockwidth,
				minimum height = 3 * \defaultblockheight]
	                        (pi){4x USB-Adapter (or Banana Pi)};
	\node [my_block, bg_fill_20,
                                above of = pi,
				minimum width = \defaultblockwidth,
				minimum height = 3 * \defaultblockheight]
				(desktop){Desktop};
	\node [my_block, bg_fill_20,
                                below left of = pi,
				minimum width = \defaultblockwidth,
 				minimum height = 3 * \defaultblockheight]
  	                        (notebook1){Notebook};
       	\node [my_block, bg_fill_20,
                                below right of = pi,
				minimum width = \defaultblockwidth,
				minimum height = 3 * \defaultblockheight]
				(notebook2){Notebook};
	\draw[<->] (pi) -- (desktop) node[midway,above,sloped] {Control (USB)};
	\draw[<->] (pi) -- (notebook1) node[midway,above,sloped] {Ethernet};
	\draw[<->] (pi) -- (notebook2) node[midway,above,sloped] {Ethernet};
\end{tikzpicture}
 \caption{Suggested network topology.}
  \label{fig:setup}
\end{figure}

\subsection{My (new) code does not seem to work at all. I've been ignoring the {\em clock skew} warning from {\tt make}. What is going on?}

The PI does not have a real-time clock. Thus, after a reboot, it can
find itself in the past. Your latest edits receive time stamps that
pre-date the binaries it created in the past before its time travels.
Make builds programs using timestamps. Thus, {\tt make} did NOT rebuild
your code, but it figured something might be fishy because your binaries
are from the future. As your binary is your old binary, your new features
do not work at all.

{\bf Solution:} If you ever get a {\em clock skew} warning, run {\tt
  make clean} to delete all old binaries.


\subsection{My notebook cannot ping the desktop}

This is normal, as your switch/router logic does not forward the
traffic to the desktop's host operating system.  Your notebooks can
only reach systems that {\em your} hub, switch or router allows them
to reach.


\subsection{How do I debug my code {\em without} {\tt fprintf()}?}

First, make sure you compiled your code with debug symbols and
without optimizations ({\tt gcc -g -O0}).  Then, launch your program as usual:
\begin{verbatim}
# ./network-driver IFCs - prog ARGs
\end{verbatim}
Then, in another shell, check with {\tt ps} which PID your ``prog'' process has:
\begin{verbatim}
# ps ax | grep prog
\end{verbatim}
Now you can attach {\tt gdb} to your running program (substitute \$PID for the PID you got from {\tt ps}):
\begin{verbatim}
# gdb prog
(gdb) attach $PID
\end{verbatim}
You should now set a breakpoint at a location where you want to start debugging,
and then continue execution until the breakpoint is reached:
\begin{verbatim}
(gdb) break handle_frame
(gdb) continue
\end{verbatim}
Then, use ``CTRL-x a'' to enable GDB's code inspection mode to see where you are
executing.  Use ``n(ext)'' to execute the next statement. Use ``s(tep)'' to step
into a function call.  Use ``print'' or ``x'' (eXamine) to inspect variables and
memory.  Use ``cont(inue)'' to run until the next breakpoint. Use ``q(uit)'' to
exit {\tt gdb}. Note that your process will continue.

Final note: this only works if your program does not crash before you can attach
{\tt gdb}.  Enable core dumps and inspect the crash with gdb if your program
crashes:
\begin{verbatim}
# ulimit -c 999999
# echo core > /proc/sys/kernel/core_pattern
# ./network-driver ... # reproduce crash
# gdb prog core
(gdb) bt full # view stack trace at time of crash
\end{verbatim}
Read the {\tt gdb} manual for more information on how to use the GNU Debugger!


\subsection{I get not network traffic on the interface. Why?}

The most common cause is that you did not enable the network adapter,
so the interface is still physically down. For each network interface,
run:
\begin{verbatim}
# ip link set up dev INTERFACENAME
\end{verbatim}
You can check that it worked using:
\begin{verbatim}
# ip link list
\end{verbatim}
If you still see a ``DOWN'' in the line, check the cable.

Your USB port may also supply insufficient power to the USB adapter.
In this case, try another port.


\section{Protocol details}

\subsection{MTU $=$ 1500 or 1514?}

When we traditionally speak of an MTU of 1500 for Ethernet, this
{\em excludes} the Ethernet header (and CRC, and preamble).  When
reporting the MTU in an ICMP ``fragmentation needed'' message,
the Ethernet header is also excluded (layer 3!).

Note that the MTU command-line argument (!) for the {\tt router} {\em
  excludes} the Ethernet header, but {\bf internally} the \texttt{mtu}
member of the {\tt struct Interface} {em includes} the Ethernet header
(see check in {\tt forward\_to}).  So you must subtract the size of
the Ethernet header when generating the ICMP message!


\end{document}

% Legacy issues, should be resolved.

\subsection{I can't compile the ''network-driver'' on a (lab) PC}

When I try to compile the ''network-driver'' on a lab PC booted from the USB-Stick (Ubuntu 16.04.5 LTS) I get the following error:

\begin{verbatim}
/usr/include/linux/llc.h:26:27: error: ‘IFHWADDRLEN’ undeclared here (not in a function)
\end{verbatim}

Add the line
\begin{verbatim}
#include <linux/if.h>
\end{verbatim}

after the line
\begin{verbatim}
#include <linux/socket.h>
\end{verbatim}

of the file ''/usr/include/linux/llc.h''


\subsection{When running the network-driver with the hub it terminates with ''write-error to tun: Message too long''}

The problem occures because the ethernet card/driver uses ''generic receive offload'' (gro). You have to disable ''gro'' on all interfaces handled by the ''network-driver'' (not only for use with the hub).\newline

To disable ''gro'' on e.g. {\tt eth0} use (as root):

\begin{verbatim}
# ethtool -K eth0 gro off
\end{verbatim}

To display the actual settings use:

\begin{verbatim}
# ethtool -k eth0
\end{verbatim}

To reenable ''gro'' use (as root):

\begin{verbatim}
# ethtool -K eth0 gro on
\end{verbatim}

Eventualy you must also disable ''generic segmentation offload'' (gso).\newline
Details about ''gro'' and ''gso'' see:
\begin{itemize}
\item Wikipedia about
\href{https://en.wikipedia.org/wiki/Large_receive_offload}
{''Large Receive Offload''}\newline
\url{https://en.wikipedia.org/wiki/Large_receive_offload}
\item LWN article \href{https://lwn.net/Articles/358910/}{''JLS2009: Generic receive offload''}\newline
\url{https://lwn.net/Articles/358910/}
\end{itemize}
